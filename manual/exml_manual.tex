%% -*- encoding: utf-8 -*-
\documentclass[a4paper,11pt]{report}
\usepackage{url}
\usepackage{times}
\usepackage[utf8]{inputenc}

\title{ExportXMLv2: description of XML file format and Java library}
\author{Yannick Versley}
\date{}

\begin{document}
\maketitle

\chapter{The EXML file format}
The ExportXMLv2 file format consists of two parts: a \textbf{schema} section,
which describes the types of objects used as terminals an markables, and
a \textbf{body} section that contains a document with annotations according to
that schema.

\section{Data Model}
A document consists of the following kinds of objects:

\begin{itemize}
\item A \textbf{terminal} represents one token in a document and the annotations
directly pertaining to that token (part of speech, lemma, dependency, \dots).
In an XML file, terminals have an \texttt{xml:id} property which uniquely identifies
them. Text spans are expressed in terms of terminal IDs (usually in the form of
\texttt{word\_2..word\_3}, references to terminals also use its ID.
\item A \textbf{markable} represents an annotated span of text (a sentence, syntactic phrase,
discourse EDU, \dots). In an XML file, markables have an \texttt{xml:id} property
that uinquely identifies them (in references from other markables or terminals).
\item An \textbf{edge} is associated to a terminal or markable, and contains
some attributes of its own. Edges cannot be referenced and do not have an ID property.
\end{itemize}


\section{The Schema}
The schema part of the XML file (enclosed with \texttt{<schema>} tags)
contains information describing the object types used in an EXML document
and the attributes it contains.

\begin{itemize}
\item A \texttt{tnode} element describes the type of terminals with it attributes.
The terminal type \emph{must} be called \emph{word} and the \texttt{tnode}
must have a \texttt{name="word"} XML attribute.
\item a \texttt{node} element describes the type of the markables contained
in one markable level (the one mentioned in the \texttt{name} XML attribute),
with its object attributes.
\item a \texttt{edge} element describes the attributes that an edge has,
and the markables/terminals it can occur with (using the \texttt{parent}
XML attribute).
\end{itemize}

All three kinds of objects can have attributes, which in turn can be of
several kinds:
\begin{itemize}
\item \textbf{String attributes} are introduced by a \texttt{text-attr}
node in the schema and can contain an arbitrary string.
\item \textbf{Enumeration attributes} are introduced by a \texttt{enum-attr}
node in the schema. An object attribute of the enumeration type can contain
one of a fixed number of values. The declaration of an enumeration attribute
can list the possible values in the enumeration (and some explanation for
that value) with \texttt{val} elements inside the \texttt{enum-attr} declaration.
\item \textbf{References} are introduced by a \texttt{node-ref} node in the schema.
The strings that these attributes contain in the XML file are translated to the
objects (markable or terminal) that they reference through the use of their \texttt{xml:id}
attribute.
\end{itemize}

\section{The Document Body}
The document body is contained in the XML file, enclosed with a \texttt{body} tag.
The terminals and markables of the document (together with any edges attached to them)
are contained inside that XML section, and each terminal, markable and edge
has one XML tag with its attributes.

The order of opening and closing tags is as follows:
\begin{itemize}
\item The tags corresponding to \textbf{terminals} are ordered
as they occur in the document (i.e., in reading order). The XML
tag for a terminal is empty except for the edges that belong to it.
\item The tags corresponding to \textbf{markables} start at the
position where their span starts. They either (indirectly or directly)
contain all the terminals that are in their span, or they are serialized
in \emph{truncated} form, where their XML span is only a part of their actual
span. For markables that are truncated in the serialization or those
that contain holes, a \texttt{span} attribute contains the actual span
for the markable.
\item The tags corresponding to an \textbf{edge} starts within the tag for the
object (terminal or markable) that it belongs to, with no tags corresponding
to markables or terminal nodes intervening (i.e., directly after the parent
object's starting tag, except for other edges).
\end{itemize}

\section{Hierarchy Order}
In ordering the XML tags for different objects (with the same span
or with overlapping spans), the EXML format respects a \emph{hierarchy order},
a partial ordering of objects:

\begin{itemize}
\item Terminals are always \emph{inside} the markables that they overlap.
Markables are never \emph{inside} another.
\item A markable with a span that is a sub-span of another markable's is
\emph{inside} that other markable.
\item A markable level can be declared with a \texttt{locality}, by
putting the name of the locality markable level in the \texttt{locality}
attribute in the declaring \texttt{node} tag. A markable on a given markable
level is always \emph{inside} the smallest enclosing markable on the level's
locality markable layer. (Example: When a \texttt{sentence} markable and
a \texttt{node} syntax markable have the same span, the \texttt{sentence}
markable is the outer one because \emph{sentence} is the locality of
\emph{node}).
\item A markable layer can have a \texttt{node-ref} declaration with a
\texttt{restriction="up"} attribute. In that case, a markable is \emph{inside}
the target of that reference. If there is a \texttt{node-ref} declaration
with a \texttt{restricion="down"} attribute, the target of the reference is \emph{inside}
the markable that (directly or indirectly) has the reference.\footnote{The \texttt{parent}
attribute for syntax nonterminals would be good candidates for \texttt{restriction="up"}.
Currently they are not declared as such.}
\end{itemize}

\section{An Example Document}
Here's a small example document illustrating some of the principles mentioned earlier:

\begin{verbatim}
<?xml version="1.0"?>
<exml-doc>
<schema>
<tnode name="word">
  <text-attr name="form"/>
</tnode>
<node name="A">
<text-attr name="B"/>
</node>
</schema>
\end{verbatim}

In the first part of the document, two kinds of objects are described: terminals that have the
type name ``\emph{word}'', with an attribute named \emph{form} 
and one kind of markables that has the name ``\emph{A}'', with
an attribute called ``\emph{B}''.

\begin{verbatim}
<body serialization="inline">
<A xml:id="a1" span="t1..t3" B="42">
<word xml:id="t1" form="super"/>
</A>
<A xml:id="a2">
<word xml:id="t2" form="cali"/>
<word xml:id="t3" form="fragilisti"/>
<word xml:id="t4" form="expi"/>
</A>
<word xml:id="t5" form="aligetic"/>
</body>
\end{verbatim}
There are two \texttt{A} markables. The first one is truncated and contains an
explicit span from \texttt{t1} to \texttt{t3}. The second one is not truncated
and has an implicit span corresponding to the enclosed terminals -- i.e., from
\texttt{t2} to \texttt{t4}.

\chapter{The Java Library}
The Java library constructs an in-memory representation from the form in an EXML
file. In this, a \texttt{Document} object contains a series of terminals (usually
as \texttt{GenericTerminal} objects) and one or several \texttt{MarkableLevel}
objects that allow to access the markables of one particular level (e.g.,
all \texttt{node} objects from the syntax layer, or all \texttt{ne}
objects from the named entity layer).

\section{The Document object}
Document objects are declared in the class \texttt{exml.Document}. The static method
\texttt{createDocument} allows to create a new document with \texttt{GenericTerminal}
terminals.

A document contains a list of terminals and a mapping of object IDs (the \texttt{xml:id}
attributes in the XML file) to actual objects.
\begin{itemize}
\item \texttt{getTerminal(int idx)} returns the terminal at the corresponding
positions. Terminals are numbered from 0 on in reading order.
\item \texttt{getTerminals()} returns a list with all terminals.
\item \texttt{getTerminalsWithEdge(String relName)} returns all terminals
that have a \texttt{relName} edge. (Example: connective or word sense annotations).
\item \texttt{terminalSchema()} returns the object schema for terminals.
\item \texttt{getAttributes(String[] attributeNames, int idStart, int idEnd)} is
a convenience function that returns a table (array of arrays) with the attributes from
\texttt{attributeNames} in the terminal range from \texttt{idStart} to \texttt{idEnd} (inclusive).
\item \texttt{markableLevelByName(String name, boolean create)} returns the markable level
with the given name. If it does not exist and \texttt{create} is true, a new markable
level is created.
\end{itemize}

\section{MarkableLevel}
A given \texttt{MarkableLevel} object manages all the markables for one given markable
level (e.g., all named entity annotations). The simplest method to get hold of a new markable
level is to call \texttt{markableLevelByName} on a document with \texttt{create=true}.
\end{document}

